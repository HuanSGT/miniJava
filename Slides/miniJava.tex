\def\pgfsysdriver{pgfsys-dvipdfm.def}
\pdfpagewidth=\paperwidth
\pdfpageheight=\paperheight

\documentclass[12pt]{beamer}
%\usetheme{Warsaw}

\usepackage{color}
\usepackage{xcolor}
\usepackage{indentfirst}
\usepackage[export]{adjustbox}
\usepackage{float}
\usepackage{wrapfig}
\usepackage{setspace}
\usepackage{fontspec}
\usepackage{graphicx}
\usepackage{subcaption}
\usepackage{pgfpages}
\usepackage{amsmath}
\usepackage{bbm}
\usepackage{dsfont}
\usepackage{xeCJK}
\usepackage{minted}
\usepackage[backend=biber]{biblatex}

%\setbeamertemplate{footline}{}
\setbeamertemplate{headline}{}
\setbeamertemplate{navigation symbols}{}
%\setbeameroption{show notes on second screen=right}
\setbeamertemplate{note page}[plain]

\addtobeamertemplate{frametitle}{\vspace*{-5pt}}{\vspace*{0pt}}

\renewcommand*{\thefootnote}{[\arabic{footnote}]}

\usefonttheme{professionalfonts}

\setsansfont{Open Sans}[Scale=0.8]
\setCJKsansfont[Scale=0.8]{WenQuanYi Micro Hei}

%\setbeamerfont{title}{ size={\fontsize{16}{16}} }
%\setbeamerfont{date}{ size={\fontsize{20}{20}} }
%\setbeamerfont{note page}{	size={\fontsize{14}{14}} }
%\setbeamerfont{note title}{ size = {\fontsize{12}{12}} }
%\setbeamerfont{institute}{ size={\fontsize{7}{7}} }
%\setbeamerfont{footnote}{ size={\fontsize{7}{7}} }
%\setbeamerfont{footline}{ size={\fontsize{7}{7}} }

\newcommand{\codesize}{\fontsize{7.7}{7.7}}


\title{}

\author{}

\date{}

\newcommand{\insfig}[2][1]{
	\begin{figure}
		\includegraphics[width=#1\textwidth]{#2.pdf}
	\end{figure}
}
\newcommand{\hs}[1]{
	\begin{minted}[fontsize=\tiny]{haskell}
	#1
	\end{minted}
}

\begin{document}

\begin{spacing}{1.3}

\begin{frame}[fragile=singleslide]{The Parser Type}

\begin{minted}[fontsize=\codesize]{haskell}
newtype Parser a = Parser { parse :: String -> [(a,String)] }

instance Monad Parser where
    return a = Parser ( \cs -> [(a,cs)] )
    p >>= f = Parser ( \cs -> do { (a,cs') <- parse p cs; parse (f a) cs' } )

(+++) :: Parser a -> Parser a -> Parser a
p +++ q = Parser (\cs -> case parse p cs ++ parse q cs of
                 [] -> []
                 x:xs -> [x] )

zero :: Parser a
zero = Parser ( \cs -> [] )
\end{minted}
\end{frame}

\begin{frame}[fragile=singleslide]{Combinators - Char and String}

\begin{minted}[fontsize=\codesize]{haskell}
item :: Parser Char
item = Parser f
    where f [] = []
          f (c:cs) = [(c,cs)]

sat :: (Char -> Bool) -> Parser Char
sat p = do { c <- item; if p c then return c else zero }

sat' :: (Char -> Bool) -> Parser String
sat' p = do { x <- sat p; return [x] }

char :: Char -> Parser Char
char c = sat (c==)

string :: String -> Parser String
string "" = return ""
string (c:cs) = char c >> string cs >> return (c:cs)

next :: String -> Parser String
next cs = string cs +++ (item >> next cs)
\end{minted}
\end{frame}

\begin{frame}[fragile=singleslide]{Combinators - Applying a Parser}

\begin{minted}[fontsize=\codesize]{haskell}
space :: Parser String
space = sat' Char.isSpace

comments :: Parser String
comments = (string "/*" >> next "*/") +++ (string "//" >> next "\n")

white :: Parser String
white = asterisk (space +++ comments) >>= return . concat

token :: Parser a -> Parser a
token p = do { a <- p; white; return a }

apply :: Parser a -> String -> [(a,String)]
apply p = parse (white >> p)
\end{minted}
\end{frame}

\begin{frame}[fragile=singleslide]{Combinators - Parsing Sequence}
\begin{minted}[fontsize=\codesize]{haskell}
asterisk :: Parser a -> Parser [a]
asterisk p = plusSign p +++ return []

plusSign :: Parser a -> Parser [a]
plusSign p = do { x <- p; xs <- asterisk p; return (x:xs) }

-- ( (a op a) op a ) op a
lass :: Parser a -> Parser op -> (a -> b) -> (b -> op -> a -> b) -> Parser b
lass a op single cons = a >>= rest . single
    where rest x = ( do
                   y <- op
                   z <- a
                   rest $ cons x y z ) +++ return x

-- a op ( a op (a op a) )
rass :: Parser a -> Parser op -> (a -> b) -> (a -> op -> b -> b) -> Parser b
rass a op single cons = a >>= rest
    where rest x = ( do
                   y <- op
                   z <- a
                   r <- rest z
                   return $ (cons x y r) ) +++ return (single x)
\end{minted}
%\hspace{-5pt}
\end{frame}

\end{spacing}
\end{document}