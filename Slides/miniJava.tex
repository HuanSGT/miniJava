\def\pgfsysdriver{pgfsys-dvipdfm.def}
\pdfpagewidth=\paperwidth
\pdfpageheight=\paperheight

\documentclass[12pt]{beamer}
\usetheme{Warsaw}

\usepackage{color}
\usepackage{xcolor}
\usepackage{indentfirst}
\usepackage[export]{adjustbox}
\usepackage{float}
\usepackage{wrapfig}
\usepackage{setspace}
\usepackage{fontspec}
\usepackage{graphicx}
\usepackage{subcaption}
\usepackage{pgfpages}
\usepackage{amsmath}
\usepackage{bbm}
\usepackage{dsfont}
\usepackage{xeCJK}
\usepackage{minted}
%\usemintedstyle{fruity}
\usepackage[backend=biber]{biblatex}
%\usecolortheme{albatross}

\setbeamertemplate{footline}{}
\setbeamertemplate{headline}{}
\setbeamertemplate{navigation symbols}{}
%\setbeameroption{show notes on second screen=right}
\setbeamertemplate{note page}[plain]

\addtobeamertemplate{frametitle}{\vspace*{-5pt}}{\vspace*{0pt}}

\renewcommand*{\thefootnote}{[\arabic{footnote}]}

\usefonttheme{professionalfonts}

\setsansfont{Open Sans}[Scale=0.8]
\setCJKsansfont[Scale=0.8]{WenQuanYi Micro Hei}

%\setbeamerfont{block body}{ size={\fontsize{10}{10}} }
%\setbeamerfont{title}{ size={\fontsize{20}{20}} }
%\setbeamerfont{date}{ size={\fontsize{20}{20}} }
%\setbeamerfont{note page}{	size={\fontsize{14}{14}} }
%\setbeamerfont{note title}{ size = {\fontsize{12}{12}} }
%\setbeamerfont{institute}{ size={\fontsize{7}{7}} }
\setbeamerfont{footnote}{ size={\fontsize{9}{9}} }
%\setbeamerfont{footline}{ size={\fontsize{7}{7}} }

\newcommand{\codesize}{\fontsize{7.7}{7.7}}

\newcommand{\insfig}[2][1]{
	\begin{figure}
		\includegraphics[width=#1\textwidth]{#2.pdf}
	\end{figure}
}
\newcommand{\hs}[1]{
	\begin{minted}[fontsize=\tiny]{haskell}
	#1
	\end{minted}
}

\title{A Monadic Parser and Interpreter for MiniJava Language}
\author{\quad\\ 李寰\inst{1} \and 章瀚元\inst{2}}

\date{} %\quad x \vspace{-50pt}}

\institute{
	\inst{1}
	%计算机科学技术学院\\
	16210240012@fudan.edu.cn
	\and
	\inst{2}
	%计算机科学技术学院\\
	16210240024@fudan.edu.cn
}

\begin{document}

\begin{spacing}{1.3}

\begin{frame}
	\titlepage
\end{frame}

\begin{frame}{Project Overview}
\begin{itemize}
\item MiniJava language
	\begin{itemize}
		\item A subset of Java
		\item Can be compiled by a regular Java compiler
		\item Basic features
			\begin{itemize}
				\item Variables assignment
				\item arithmetic operation
				\item If, while, recursion, functions invoking each other
				%\item Object-oriented feature not included
			\end{itemize}
	\end{itemize}
\item 分工
	\begin{itemize}
	\item Parser\footnote[frame]{Graham Hutton, Erik Meijer. Monadic parsing in Haskell. Journal of functional programming, 1998, 8(04): 437-444.\vspace{1pt}}: 李寰
	\item Interpreter: 章瀚元
	\end{itemize}
\end{itemize}
\end{frame}

\begin{frame}[fragile=singleslide]{The Type Parser}

\begin{minted}[fontsize=\codesize]{haskell}
newtype Parser a = Parser { parse :: String -> [(a,String)] }

instance Monad Parser where
    return a = Parser ( \cs -> [(a,cs)] )
    p >>= f = Parser ( \cs -> do { (a,cs') <- parse p cs; parse (f a) cs' } )

(+++) :: Parser a -> Parser a -> Parser a
p +++ q = Parser (\cs -> case parse p cs ++ parse q cs of
                 [] -> []
                 x:xs -> [x] )

zero :: Parser a
zero = Parser ( \cs -> [] )
\end{minted}
\end{frame}

\begin{frame}[fragile=singleslide]{Combinators - Char and String}

\begin{minted}[fontsize=\codesize]{haskell}
item :: Parser Char
item = Parser f
    where f [] = []
          f (c:cs) = [(c,cs)]

sat :: (Char -> Bool) -> Parser Char
sat p = do { c <- item; if p c then return c else zero }

sat' :: (Char -> Bool) -> Parser String
sat' p = do { x <- sat p; return [x] }

char :: Char -> Parser Char
char c = sat (c==)

string :: String -> Parser String
string "" = return ""
string (c:cs) = char c >> string cs >> return (c:cs)

next :: String -> Parser String
next cs = string cs +++ (item >> next cs)
\end{minted}
\end{frame}

\begin{frame}[fragile=singleslide]{Combinators - Applying a Parser}

\begin{minted}[fontsize=\codesize]{haskell}
asterisk :: Parser a -> Parser [a]
asterisk p = plusSign p +++ return []

plusSign :: Parser a -> Parser [a]
plusSign p = do { x <- p; xs <- asterisk p; return (x:xs) }

space :: Parser String
space = sat' Char.isSpace

comments :: Parser String
comments = (string "/*" >> next "*/") +++ (string "//" >> next "\n")

white :: Parser String
white = asterisk (space +++ comments) >>= return . concat

token :: Parser a -> Parser a
token p = do { a <- p; white; return a }

apply :: Parser a -> String -> [(a,String)]
apply p = parse (white >> p)
\end{minted}
\end{frame}

\begin{frame}[fragile=singleslide]{Combinators - Parsing Sequence}
\begin{minted}[fontsize=\codesize]{haskell}

-- ( (a op a) op a ) op a
lass :: Parser a -> Parser op -> (a -> b) -> (b -> op -> a -> b) -> Parser b
lass a op single cons = a >>= rest . single
    where rest x = ( do
                   y <- op
                   z <- a
                   rest $ cons x y z ) +++ return x

-- a op ( a op (a op a) )
rass :: Parser a -> Parser op -> (a -> b) -> (a -> op -> b -> b) -> Parser b
rass a op single cons = a >>= rest
    where rest x = ( do
                   y <- op
                   z <- a
                   r <- rest z
                   return $ (cons x y r) ) +++ return (single x)
\end{minted}
\end{frame}

\begin{frame}[fragile=singleslide]{Parsing Arithmetic Expressions - Operators and Ints}
\begin{minted}[fontsize=\codesize]{haskell}
data AddOp    = Plus | Minus

plus     = token $ string "+"  >> return Plus
minus    = token $ string "-"  >> return Minus
addOp    = plus +++ minus

data MulOp    = Times | Slash | Modulo

times    = token $ string "*"  >> return Times
slash    = token $ string "/"  >> return Slash
modulo   = token $ string "%"  >> return Modulo
mulOp    = times +++ slash +++ modulo

data BasicExpr = Bool Bool | Int Int | Nul | ...

int = token $ plusSign (sat Char.isDigit) >>= return . Int . read
\end{minted}
\end{frame}

\begin{frame}[fragile=singleslide]{Parsing Arithmetic Expressions - Nodes in AST}
\begin{minted}[fontsize=\codesize]{haskell}
data AddExpr   = AMul MulExpr
               | AddExpr AddExpr AddOp MulExpr

addExpr = lass mulExpr addOp AMul AddExpr

data MulExpr   = AUnary UnaryExpr
               | MulExpr MulExpr MulOp UnaryExpr

mulExpr = lass unaryExpr mulOp AUnary MulExpr
\end{minted}
\end{frame}

\begin{frame}[fragile=singleslide]{Parsing Arithmetic Expressions - An Example}
\begin{block}{Terminal}
\begin{minted}[fontsize=\codesize]{haskell}
ghci> fst . head $ apply addExpr "10 / 3 - 4 * 5 % 7"
(AddExpr
    (MulExpr
        (int 10)
        (op "/")
        (int 3)
    )
    (op "-")
    (MulExpr
        (MulExpr
            (int 4)
            (op "*")
            (int 5)
        )
        (op "%")
        (int 7)
    )
)
\end{minted}
\end{block}
\end{frame}

\begin{frame}[fragile=singleslide]{The Type Interp}
\begin{minted}[fontsize=\codesize]{haskell}
newtype Env = Env (Map.Map Ident Int, Map.Map Ident Func)

newtype Interp a = Interp {
    interp :: [Env] -> String -> Either (a,[Env],String) (Int,[Env],String) }

instance Monad Interp where
    return a = Interp ( \es0 cs0 -> Left (a,es0,cs0) )
    i >>= h = Interp ( \es0 cs0 -> case interp i es0 cs0 of
                                       Left  (a,es,cs) -> interp (h a) es cs
                                       Right (a,es,cs) -> Right  (a,es,cs) )
                                       
return_ :: Int -> Interp a
return_ a = Interp ( \es cs -> Right (a,es,cs) )
\end{minted}
\end{frame}

\begin{frame}[fragile=singleslide]{Interpreting Arithmetic Expressions}
\begin{minted}[fontsize=\codesize]{haskell}
addExpr :: AddExpr -> Interp Int
addExpr (AMul a) = mulExpr a
addExpr (AddExpr a b c) = do
    i <- addExpr a
    j <- mulExpr c
    case b of
         Plus  -> return $ i + j
         Minus -> return $ i - j

mulExpr :: MulExpr -> Interp Int
mulExpr (AUnary a) = unaryExpr a
mulExpr (MulExpr a b c) = do
    i <- mulExpr a
    j <- unaryExpr c
    case b of
         Times  -> return $ i * j
         Slash  -> return $ i `div` j
         Modulo -> return $ i `mod` j
\end{minted}
\end{frame}

\begin{frame}[fragile=singleslide]{Interpreting Functions}
\begin{minted}[fontsize=\codesize]{haskell}
data Func = Funci { funci :: Interp Int }
          | Funcf { funcf :: Int -> Interp Func }

env0 = [] :: [Env]
env1 = Env (Map.empty,Map.empty) :: Env

closure :: Interp Int -> Interp Int
closure i = Interp (\es0 cs0 -> case interp i (env1:es0) cs0 of
                                   Right (a,e:es,cs) -> Left (a,es,cs)
                                   Left  (a,e:es,cs) -> Left (a,es,cs) )
                  
valf :: Ident -> Interp Func

apply :: [Int] -> Interp Func -> Interp Int

invoke :: [Int] -> Ident -> Interp Int
invoke xs id = closure . apply xs . valf $ id
\end{minted}
\end{frame}

\iffalse
\begin{frame}[fragile=singleslide]{Interpreting Functions}
\begin{minted}[fontsize=\codesize]{haskell}
func :: Maybe Paras -> Block -> Func
func aa b =
    case aa of
         Nothing -> Funci $ block b
         Just (APara (Para _ id)) -> Funcf $ \x -> (exti id x) >> h
             where h = return $ func Nothing b
         Just (Paras (Para _ id) _ p) -> Funcf $ \x -> (exti id x) >> h
             where h = return $ func (Just p) b

eval :: Program -> Interp Int
eval (Program [ClassDecl a b c d es f] EOF) = closure h
    where h = methods es >> invoke [] (Ident "main")
              where methods ...
\end{minted}
\end{frame}
\fi

\end{spacing}
\end{document}